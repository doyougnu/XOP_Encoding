% Created 2017-03-05 Sun 18:58
% Intended LaTeX compiler: pdflatex
\documentclass[10pt, letterpaper]{article}
\usepackage[utf8]{inputenc}
\usepackage[T1]{fontenc}
\usepackage{graphicx}
\usepackage{grffile}
\usepackage{longtable}
\usepackage{wrapfig}
\usepackage{rotating}
\usepackage[normalem]{ulem}
\usepackage{amsmath}
\usepackage{textcomp}
\usepackage{amssymb}
\usepackage{capt-of}
\usepackage{hyperref}
\usepackage[margin=1in]{geometry}
\usepackage{adjustbox}
\author{Jeffrey Young}
\date{Feb 23, 2017}
\title{Coding Notes}
\hypersetup{
 pdfauthor={Jeffrey Young},
 pdftitle={Coding Notes},
 pdfkeywords={},
 pdfsubject={},
 pdfcreator={Emacs 26.0.50 (Org mode 9.0.5)}, 
 pdflang={English}}
\begin{document}

\maketitle

\section*{Purpose}
\label{sec:org4e0034f}
The Purpose of encodings is to systematically and formally codify real world, in the wild, explanations so that observing larger patterns becomes possible. The end goal of this is:
\begin{enumerate}
\item Generate a database of encodings
\item Analyze database to find patterns of explanation
\item 1 and 2 become formative work for exploring DSL possibilities in XOP
\end{enumerate}
\section*{Scope}
\label{sec:org0d4f588}
The scope of the database is restricted to explanations of common Computer Science algorithms \uline{from} Universities only. Restricting the scope in this manner provides two benefits:
\begin{enumerate}
\item All explanatory objects have a stated, intrinsic goal to communicate the mechanics, application, and implementation of similar things
\item There are numerous examples of different approaches to explain \uline{the same} thing, and numerous examples of \uline{like} approaches to explain different things
\end{enumerate}

\section*{Some Terminology}
\label{sec:orgbe0d9dc}
In our terminology an explanation is called an "explanation artifact", our
  working model of the process of explaining is a non-linear sequence of
  "steps", where each "step" denotes some progress in the understanding of the
  explanation artifact on the part of the information receiver. More formally:
\begin{description}
\item[{Explanatory Artifact}] The whole explanation, including all the step taken in the explanation, the language used in the explanation etc.
\item[{Step}] The steps that are taken, in an explanatory artifact, that guide the
reader from non-understanding to understanding.
\end{description}
\section*{Syntax}
\label{sec:org1ee15f7}
The syntax of an encoding is given by \_ - \_ - \_ - \_ \(\subseteq\) Location \texttimes{} Level \texttimes{} Role \texttimes{} Notation where \\

l \(\in\) \(\mathbb{N}\) \\

g \(\in\) Level ::= Problem | Algorithm | Implementation \\

r \(\in\) Role ::= Background | Definition | Constraint | Example | Application | Variant | Analogy | Idea | Proof | Performance | Properties \\

n \(\in\) Notation ::= English | Math | Diagram | List | Table | Sequence | Pseudocode | Code | Animation | Picture | Plot | Empty | n/n \\

\section*{Semantics}
\label{sec:orgc0bd108}
\subsection*{Location}
\label{sec:org41bae11}
a location \(l\), specifies the location that the encoding is referring to,
this could be a slide, a number line etc.
\subsection*{Level}
\label{sec:org6cfb060}
a Level \(g\), Specifies the level of abstraction for a given step, a level can by one of:
\begin{description}
\item[{Problem}] \(\triangleq\) the purpose of a given step is to elucidate the
motivating problem of the algorithm, i.e the problem that the algorithm
will solve
\item[{Algorithm}] \(\triangleq\) the purpose of a given step is to explain the
algorithm at hand
\item[{Implementation}] \(\triangleq\) the purpose of a given step is to explain the
implementation details of the algorithm, e.g. What data structures to
use, What the form of the code that implements the algorithm should be
\end{description}

\subsection*{Role}
\label{sec:orgbce9be6}
a Role \(r\), specifies how the Level is trying to be reached, denotes the
answer to question such as "What is the step trying to convey?":

In general the meaning of each role is:
\begin{description}
\item[{Background}] \(\triangleq\) A given level is reached by a step that describes the history,
creators, genealogy of the Algorithm
\item[{Definition}] \(\triangleq\) A given level is reached by a step that explicitly provides a
formal definition.
\item[{Constraint}] \(\triangleq\) A given level is reached by a step that explicitly presents a
limit or condition in which the level would cease to be valid, e.g.
Dijkstra's only works on non-negative weighted graphs
\item[{Example}] \(\triangleq\) A given level is reached by a step that provides an Example
\item[{Application}] \(\triangleq\) A given level is reached by a step that explains what the
algorithm is useful for
\item[{Variant}] \(\triangleq\) A given level is reached by a step that describes things that are
similar but slightly different than the algorithm. For example, describing
Prim's algorithm and it's similarities to Dijkstra's or describing the
similarities between a dog and a wolf
\item[{Analogy}] \(\triangleq\) A given level is reached by a step that provides an Analogy to
explain the algorithm at hand. For example a visual analogy for Dijkstra's
could be: If you have a physical model of a graph, and you pick it up by
one vertex, then the vertex with the shortest path to the "source" vertex
will be the one farthest from the ground.
\item[{Idea}] \(\triangleq\) A given level is reached by a step that adds an abstract idea to the
explanation as a way to progress. For example, the statement "Well we have
this, \uline{what if we did} this?"
\item[{Proof}] \(\triangleq\) A given level is reached by a formal proof
\item[{Performance}] \(\triangleq\) A given level is reached by a step that explicitly describes
the computational complexity of the level
\item[{Properties}] \(\triangleq\) A given level is reached by a step that explicitly describes the properties of that level. For Example, AVL Trees are both \emph{balanced} and \emph{ordered}.
\end{description}

Consider the following matrix of Level Role combinations of Dijkstra's
algorithm. Not all of the cells will be orthogonal to each other. In this case
we would have: \\

\textbf{Problem}: How to traverse the shortest path in a non-negative
weighted graph \\

\textbf{Algorithm}: Dijkstra's Algorithm \\

\textbf{Implementation}: You should use a Priority Queue that has
efficient lookup, mutate operations. \\

\begin{itemize}
\item \(\bot\) used to denote cells which may be nonsensical \\
\end{itemize}

\begin{Table}
\adjustbox{max width=\linewidth}{
\centering
\begin{center}
\begin{tabular}{|c|c|c|c|}
Role & Problem & Algorithm & Implementation\\
\hline
Definition & Mathematical definition of Problem & Mathematical Definition of Algorithm & \bot\\
Example & Display of a non-negative weighted graph & Showing the algorithms execution on the map & Showing requisite data structures etc.\\
Application & Real World Example of the problem & \bot & Triage System in a Hospital\\
Background & History of the Problem & History, Author, etc. & History of Priority Queues\\
Variant & Perhaps a teleporter exists, now what is shortest path & Description of Bellman-Ford & Description of slightly different Priority Queues\\
Analogy & \bot & Exposition of Prim's algorithm & \bot\\
Performance & \bot & Complexity & Complexity of requisite data structs\\
Idea & \bot & \bot & \bot\\
Constraint & Depiction of the Constraints of the Problem & Depiction of domain where Algorithm lacks validity & Requirements of internal Data Structs\\
Proof & \bot & Explicit Proof of Algorithm correctness & Explicit Proof of some requisite part of the algorithm\\
\end{tabular}
\end{center}
\end{Table}

\subsection*{Notation}
\label{sec:org8a4d359}
a Notation \(n\), specifies the form of the role, and can be one of:
\begin{description}
\item[{English}] \(\triangleq\) Human language to give explanations/statements.
\item[{Diagram}] \(\triangleq\) Diagram in the manner of data structures, such as graph, list.
\item[{List}] \(\triangleq\) List of similar items
\item[{OrderedList}] \(\triangleq\) Step by step items
\item[{Math}] \(\triangleq\) Formulas/math style symbols.
\item[{Pseudocode}] \(\triangleq\) Algorithm presented as pseudocode
\item[{Code}] \(\triangleq\) executable code to show the algorithm explicitly
\item[{Table}] \(\triangleq\) Explanatory information displayed in a table
\item[{Animation}] \(\triangleq\) a gif or animation of any type is used.
\item[{Picture}] \(\triangleq\) A photo/screenshot or picture is used.
\item[{Sequence}] \(\triangleq\) A conjunction of steps meant to show progress in a serial manner
\item[{Plot}] \(\triangleq\) A mathematically generated plot that adheres to some coordinate system
\end{description}

for example a definition might be described in English, followed by the same
definition described by geometry. Notations can be combined for a single
location like so:
\begin{equation}
   \(\frac{n \in \text{Notation} \quad m \in \text{Notation}}{n/m \in \text{Notation}}\)
\end{equation}
\end{document}