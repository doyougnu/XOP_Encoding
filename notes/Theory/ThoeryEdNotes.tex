% Created 2017-03-30 Thu 16:56
% Intended LaTeX compiler: pdflatex
\documentclass[10pt, letterpaper]{article}
\usepackage[utf8]{inputenc}
\usepackage[T1]{fontenc}
\usepackage{graphicx}
\usepackage{grffile}
\usepackage{longtable}
\usepackage{wrapfig}
\usepackage{rotating}
\usepackage[normalem]{ulem}
\usepackage{amsmath}
\usepackage{textcomp}
\usepackage{amssymb}
\usepackage{capt-of}
\usepackage{hyperref}
\usepackage[margin=1in]{geometry}
\usepackage{bussproofs}
\author{Jeffrey Young}
\date{Mar 30, 2017}
\title{Primer on Educational Theoretical Underpinnings}
\hypersetup{
 pdfauthor={Jeffrey Young},
 pdftitle={Primer on Educational Theoretical Underpinnings},
 pdfkeywords={},
 pdfsubject={},
 pdfcreator={Emacs 26.0.50 (Org mode 9.0.5)}, 
 pdflang={English}}
\begin{document}

\maketitle
\bibliography{TheoryEdNotes}

\section*{Orientation}
\label{sec:org537e346}
In the Educational theory literature we are after what is called a \emph{typology of explanation} several exist and have been developed through the years see this paper \cite{brown1984explaining} for an overview.

\section*{Purpose}
\label{sec:org564d1bf}
Most of the papers I've found have developed a typology, then performed a experimental study of some sort, and then performed some statistical analysis to determine significance between some control group and some other group, typically a chi-squared test. The purpose of this document is to serve as a reference point for papers that may be interesting and to summarize those papers in terms that will be useful. I see to major roads that we can go down based on these papers: 1) We can develop our own typology, specifically designed to explicate and elucidate explanatory objects whose purpose is to explain algorithms - this is an extension of what we've already done. The end of this branch, in my mind, is a user study similar to what some researchers in this area have done with lectures (see Rosenshine below), this would allow us to say these categories are important, these arent' and so on. The other path I can envision is to recodify our methods in terms of these researchers methods, specifically in terms of Bellack et al 1966.

\section*{Papers to look at}
\label{sec:org5c22cc4}
\begin{itemize}
\item Bellack et al, The Language of the classroom \cite{bellack1966language}, look at this, Gage and Rosenshine indicate this is the source work for explanation in a pedagogical context
\end{itemize}

\section*{Papers that will be useful or relevant}
\label{sec:org03415b2}

\subsection*{Papers that detail what good explanations are}
\label{sec:org7e2f1e6}
\begin{itemize}
\item Brown and Armstrong: Explaining and Explanations \cite{brown1984explaining}
\item Smith and Meux 1970
\item Thyne 1963; 1968; The Psychology of Learning and Techniques of Teaching \cite{thyne1965psychology}
\item Turney et al., 1975
\item Gage and Berliner 1975; Research into Classroom Processes
\item Brown 1978a, 1978b; Microteaching: Programme of Teaching Skills \cite{10.2307/3120386}
\end{itemize}

\subsection*{Papers that detail different Typography's of Explanations}
\label{sec:org546c812}
\begin{itemize}
\item Swift 1967, Language and Concepts in Teaching \cite{smith1967language}
\item Bellack et al, The Language of the classroom \cite{bellack1966language}
\item Ennis 1969, Logic in Teaching \cite{ennis1969logic}
\item Smith and Meux: Study in the logic of teaching \cite{Smith1970-SMIASO-13}
\item Hyman 1974: Vantage Points for Study \cite{hyman1968teaching}
\end{itemize}


\section*{Explaining and Explanations Typology}
\label{sec:orgb505335}

\subsection*{Keys}
\label{sec:org100db2f}
An explanation is divided up into "keys":
\begin{itemize}
\item Keys \(\triangleq\) A Key is part of the main explanation that must be explained for one to understand that which is to be explained
\end{itemize}
\subsection*{Links}
\label{sec:org23584d7}
One would then seek to "link" the discrete "keys" together: \\
\begin{description}
\item[{Links}] PAPER DOES NOT DEFINE
\end{description}
\subsection*{Rules}
\label{sec:orgf29f32d}
One would also define any rules that are pertinent to what is being explained
\begin{description}
\item[{Rules}] PAPER DOES NOT DEFINE
\end{description}
\subsection*{The Typology}
\label{sec:org984eb08}
The Typology is formed by 3 categories of explanation:

\subsubsection*{The Interpretive}
\label{sec:orgf31f2db}
That which clarifies, exemplifies, or interprets the meaning of terms. Roughly speaks to "What is \ldots{} ?" 
\subsubsection*{The Descriptive}
\label{sec:org575cbf7}
That which describes a process or structure. Speaks to "How is ..? How does \ldots{}?" 
\subsubsection*{The "Reason Giving"}
\label{sec:org75f4009}
That which offers reasons or causes. Speaks to the occurrence of a phenomenon (Why is \ldots{}?) \\

\subsubsection*{An Example:}
\label{sec:org1c3a3c2}

\begin{itemize}
\item Interpretive Explanations:
\label{sec:org2867fe5}
\begin{itemize}
\item What are phyla?
\item What is a biome?
\item What is a fossil?
\item What is ecological succession?
\end{itemize}

\item Descriptive Explanations:
\label{sec:org87cd0bc}
\begin{itemize}
\item Where does the energy of the living world come from?
\item How do streams come to be polluted?
\item How do environmental factors influence the number of plants and animals in a particular way
\item How are animal protected against the dangers of drying out
\end{itemize}

\item Reason-giving Explanations:
\label{sec:org73fc097}
\begin{itemize}
\item Why are there no polar bears in the South Pole
\item Why is soil considered to be an ecosystem
\end{itemize}

I think that these roughly translate to our "Levels" in our coding scheme.
\end{itemize}

\section*{Gage et al Typology}
\label{sec:org1233146}
N.L. Gage's main programme was to identify objective variables in teaching lectures that could be used to form a \emph{standard} approach to lecturing. He compares the thrust of his work to that of physics; when physics hit a theoretical limit they often sought to surpass the limit by expanding and individuating constituents. For example, the Nucleus of an Atom is actually made up of subatomic particles, then they would seek to study the subatomic particles instead of the nucleus. This typology follows from Barak Rosenshines work See Chapter 9, pg 201, Research into Classroom Processes Westbury/Bellack. In this article Rosenshine develops the typology and then runs a user study to assess the correlative impact of each category on effective and ineffective lectures. So the typology merely serves as a way to classify, and differentiate good lectures from bad ones. This paper is meh, it doesn't define everything but the results are good although it can be hard to follow at times, especially the distinction between variables and categories - which I still don't understand.

\subsection*{Variables}
\label{sec:org4d14ec2}
Rosenshine identifies 27 variables that were verbal, non-verbal, communicated by teachers and received by students. The variables were developed from research in four general areas:
\begin{enumerate}
\item Linguistics
\item Instructional Set
\item Experimental studies of instruction
\item Multivariate Studies of the Behavioral Correlates of Teacher Effectiveness
\end{enumerate}

\subsubsection*{Linguistic Categories}
\label{sec:org7001730}
Consists of 9 "categories"
\begin{enumerate}
\item Word Length
\item Total number of relevant words
\item Length and structure of independent clause unit
\item prepositional phrases
\item readability estimate \(\triangleq\) based on a multiple-regression formula developed by Flesch
\item personal references \(\triangleq\) counts of first and second person pronouns
\item negative sentences \(\triangleq\) counts of sentences containing "not" modifying the verb, noun or some similar negation
\item passive verbs \(\triangleq\) counts of independent or dependent clauses containing passive verbs
\item awkward and fragmented sentences \(\triangleq\) counts of sentences which depart from usual sentence construction or phrases which lack a subject of a verb but add information. e.g. "Now to foreign affairs"
\end{enumerate}
\subsubsection*{Instructional Set}
\label{sec:org286d9ff}
Consists of 2 categories
\begin{enumerate}
\item Structuring Sets \(\triangleq\) set that contains variables which are words or phrases which indicate that the speaker is attempting to clarify distinctions between new and previously learned material
\item Focusing or Arousing Sets \(\triangleq\) set that contains variables which might identify phrases designed to arouse or focus attention.
\end{enumerate}
\subsubsection*{Presentational Categories}
\label{sec:orga5ddad9}
Consists of 9 categories
\begin{enumerate}
\item Use of rule-and-example Pattern
\item number of examples
\item organization of topics
\item use of enumeration
\item movement and gesture
\item breaks in speech
\item use of maps and chalkboard
\item rate of speech
\item repetition and redundancy
\end{enumerate}
\subsubsection*{Multivariate studies of teaching behaviors}
\label{sec:orgd94ad60}
Consists of 7 categories
\begin{enumerate}
\item Verbal Hostility
\item non-verbal affect
\item reference to pupil's interests
\item expansion of pupil's ideas
\item ratio of acceptance and praise to criticism
\item Conditional Words \(\triangleq\) counts of words such as "but", "however" and "although"
\item Explaining Links \(\triangleq\) prepositions and conjunctions which indicated the cause, result, means, or purposes of an event or idea.
\end{enumerate}

\subsection*{Takeaway}
\label{sec:org12e7c9b}
We could co-opt a lot of rosenshines typology for our typology, a benefit of this would be that we would be in a position to co-opt his/her research on effective lectures to effective powerpoints/papers for algorithms.

\section*{Bellack Language of the Classroom}
\label{sec:orga429eee}

\subsection*{Chapter 2: Basic System For Analysis}
\label{sec:org059aa06}

\subsubsection*{Overview}
\label{sec:org5ea2999}
To analyze classroom discourse Bellack et al. developed a "system of categories devised to describe the verbal performance of teachers and students". Protocols of classroom discourse were analyzed in terms of these categories. Basically, and probably unbeknownst to them, they developed a programming language that describes classroom discourse. Keep in mind that their data is based off of transcripts of lectures, so their coding scheme naturally reflects that.
\subsubsection*{The System}
\label{sec:org9cd25d4}
This system is heavily based on Wittgenstein's concept of a language game. The basic categories are:
\begin{itemize}
\item Pedagogical Moves \(\triangleq\) classroom discourse is viewed as a kind of "language game" (this is a technical term from Wittgenstein). The basic unit of discourse is deemed a "pedagogical move" is this system. There are four types of pedagogical moves in the system:
\begin{enumerate}
\item Structuring
\item Soliciting
\item Responding
\item Reacting
\end{enumerate}
\item Teaching Cycles \(\triangleq\) Pedagogical moves occur in classroom discourse in certain cyclical patterns and combinations that are designated teach cycles.
\item Categories of Meaning \(\triangleq\) Four functionally different types of meaning are communicated by teachers and pupils in the classroom:
\begin{enumerate}
\item Substantive with associated
\item Substantive-logical meanings
\item instructional with associated
\item instructional-logical meanings
\end{enumerate}
\end{itemize}
Coding is done from the viewpoint of the observer, with pedagogical meaning inferred from the speaker's verbal behavior. 
\subsubsection*{Grammer}
\label{sec:orgd5d94c7}
Each pedagogical move is coded as follows, I've added indices to follow the text: \\
\begin{verbatim}
Move = 1. Speaker 
       | 2. Type of Move 
       | 3. Substantive Meaning
       | 4. Substantive-Logical Meanings 
       | 5. Number of Lines in (3) or (4) 
       | 6. Instructional Meanings 
       | 7. Instructional-Logical Meanings 
       | 8. Number of lines in (6) or (7) 
\end{verbatim}

\subsubsection*{Semantics}
\label{sec:org698fd97}
\begin{itemize}
\item The Only Operator
\label{sec:orgcebac40}
Moves are syntactically conjoined into strings with the "/" operator: \\
\begin{equation}
   \(\frac{n \in \text{Move} \quad m \in \text{Move}}{n/m \in \text{Move}}\)
\end{equation}
\item An Example Sentence
\label{sec:orgdb4e1e0}
An example, coded pedagogical move is: \\
\begin{verbatim}
  T/STR/IMX/XPL/4/PRC/FAC/2
\end{verbatim}
The interpretation is as follows:
\begin{verbatim}
  T / STR / IMX / XPL /  4  / PRC / FAC / 2
 (1)/ (2) / (3) / (4) / (5) / (6) / (7) /(8)
\end{verbatim}
This translates to: A \emph{teacher} (1) makes a \emph{structuring} (2) move in which they \emph{explain} (3) something about  \emph{imports and exports} (4) for \emph{four} (5) lines of transcript and also states \emph{facts} (7) about class \emph{procedures} (6) for \emph{two} (8) lines of the transcript. 
\item Structuring (STR)
\label{sec:org72e921a}
\begin{itemize}
\item \uline{Purpose}: \\
Structuring moves function to \emph{set the context} for subsequent behavior. Think of this as identifying, setting, or initializing the state.
\item \uline{Mechanics}: This is achieved by:
\begin{enumerate}
\item Launching or Halting or Exluding interactions bewteen teacher and pupils
\item Indicating the nature of the interaction in terms of the dimensions of time, agent, activity, topic and cognitive process, regulations, reasons, and instructional aid.
\end{enumerate}
\item \uline{Response}: Structuring moves do not elicit a response, are not direct responses and are not called out by anything in particular in the classroom
\item \uline{Examples}:
\begin{enumerate}
\item T/STR: All right, getting down to it now, I think international trade, then, or international economic relations, whatever you call it, is a field of study within economics which in many cases has been unfortunately divorced from or too far divorced from domestic trade because there are great similarities, and also there are some rather distinct differences.
\end{enumerate}
\end{itemize}
\item Soliciting (SOL)
\label{sec:org50a0a74}
\begin{itemize}
\item \uline{Purpose}: \\
These are moves that tend to elicit an active verbal response, a cognitive response, or a physical response to the persons addressed.
\item \uline{Mechanics}: Soliciting moves are encoded in terms of the response expected rather than the solicitation itself
\item \uline{Examples}:
\begin{enumerate}
\item T/SOL: What are the factors of production?
\item P/SOL: May we keep our books open?
\item T/SOL: Pay attention to this!
\end{enumerate}
\end{itemize}
\item Responding (RES)
\label{sec:org29554d7}
\begin{itemize}
\item \uline{Purpose}: \\
These moves are reciprocal to soliciting moves and only occur in relation to them. They function to fulfill the expectation of soliciting moves, and as such are reflexive.
\item \uline{Mechanics}: There can be no solicitation that is not intended to elicit a response, and no response that has not been directly elicited by a solicitation
\item \uline{Examples}:
\begin{enumerate}
\item T/SOl: What are the factors of production?
P/RES: Land, Labor, and capital.
\item T/SOL: What is exchange control
P/Res: I don't know
\end{enumerate}
\end{itemize}

\item Reacting (REA)
\label{sec:org9e908da}
\begin{itemize}
\item Purpose: \\
These moves are occasioned by a structuring, soliciting, responding, or some prior move but are not directly elicited by them. These qualify the moves that preceded them, either by clarifying some point, synthesizing a new one, or expanding on a previous point.
\item \uline{Mechanics}: Preceding moves \emph{only} serve as the occasion for reaction moves.
\item \uline{Examples}:
\begin{enumerate}
\item T/REA: All right
\item T/REA: That's partly it
\item T/REA: Good. It limits specifically the number of items of one type or another which can come into this country. For example, we might decide that no more than one thousand German cars will be imported in any one calendar year. This is a specific quota which the government checks.
\end{enumerate}
\end{itemize}
\begin{itemize}
\item \uline{Special Modifiers}: When a reaction move is italicized (\emph{REA}); this denotes that the reaction is occasioned by more than a single move.
\begin{itemize}
\item \uline{Special Modifier Examples}:
\begin{enumerate}
\item T/REA: All of the instances of foreign investment that we have discussed here can be classified as either direct or portfolio types of investment.
\end{enumerate}
\end{itemize}
\end{itemize}
\end{itemize}
\end{document}