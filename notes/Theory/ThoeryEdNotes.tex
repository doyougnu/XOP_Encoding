% Created 2017-04-08 Sat 14:46
% Intended LaTeX compiler: pdflatex
\documentclass[10pt, letterpaper]{article}
\usepackage[utf8]{inputenc}
\usepackage[T1]{fontenc}
\usepackage{graphicx}
\usepackage{grffile}
\usepackage{longtable}
\usepackage{wrapfig}
\usepackage{rotating}
\usepackage[normalem]{ulem}
\usepackage{amsmath}
\usepackage{textcomp}
\usepackage{amssymb}
\usepackage{capt-of}
\usepackage{hyperref}
\usepackage[margin=1in]{geometry}
\usepackage{bussproofs}
\author{Jeffrey Young}
\date{Mar 30, 2017}
\title{Primer on Educational Theoretical Underpinnings}
\hypersetup{
 pdfauthor={Jeffrey Young},
 pdftitle={Primer on Educational Theoretical Underpinnings},
 pdfkeywords={},
 pdfsubject={},
 pdfcreator={Emacs 26.0.50 (Org mode 9.0.5)}, 
 pdflang={English}}
\begin{document}

\maketitle

\section*{Orientation}
\label{sec:org5a0ec8e}
In the Educational theory literature we are after what is called a \emph{typology of explanation} several exist and have been developed through the years see this paper \cite{brown1984explaining} for an overview. This document serves as a one stop shop of sorts for useful papers, summaries or primers on said papers. If you don't have time I suggest just reading the Purpose section and the section on Bellack.
\section*{Executive Summary}
\label{sec:org0e31e8c}
We want to be focusing on Bellack's work for sure. In their research they develop a coding system, that describes qualities of in classroom lectures, and then they use that coding system to determine \emph{teaching cycles} which are patterns they noticed that arose from the data. We can and should co-opt their coding strategies and recode our data, this may also open the door to analyzing verbal lectures. What you find in this document is a summary of chapter two, where they layout the system. I'm working on reformulating our system in terms of theirs I'll post it when its finished. The big question in my mind is how this will help or connect to a DSL.

\section*{Purpose}
\label{sec:org461ffc0}
Most of the papers I've found have developed a typology, then performed a experimental study of some sort, and then performed some statistical analysis to determine significance between some control group and some other group, typically a chi-squared test. The purpose of this document is to serve as a reference point for papers that may be interesting and to summarize those papers in terms that will be useful. I see two major roads that we can go down based on these papers: 1) We can develop our own typology, specifically designed to explicate and elucidate explanatory objects whose purpose is to explain algorithms - this is an extension of what we've already done. The end of this branch, in my mind, is a user study similar to what some researchers in this area have done with lectures (see Rosenshine below), this would allow us to say these categories are important, these arent' and so on. The other path I can envision is to recodify our methods in terms of these researchers methods, specifically in terms of Bellack et al 1966. I think the latter route (a la Bellack) is the most promising, I can envision writing two papers regarding their work 1) Appropriating their system to our needs and show that it can aid in the creation of a DSL or is useful for computer science educators 2) Using the system to encode documents and crunching the data much like we were doing.

\section*{Papers that will be useful or relevant}
\label{sec:org38d88ac}

\subsection*{Papers that detail what good explanations are}
\label{sec:org33cfddb}
\begin{itemize}
\item Brown and Armstrong: Explaining and Explanations \cite{brown1984explaining}
\item Smith and Meux 1970
\item Thyne 1963; 1968; The Psychology of Learning and Techniques of Teaching \cite{thyne1965psychology}
\item Turney et al., 1975
\item Gage and Berliner 1975; Research into Classroom Processes
\item Brown 1978a, 1978b; Microteaching: Programme of Teaching Skills \cite{10.2307/3120386}
\end{itemize}

\subsection*{Papers that detail different Typography's of Explanations}
\label{sec:org42d7aa1}
\begin{itemize}
\item Swift 1967, Language and Concepts in Teaching \cite{smith1967language}
\item Bellack et al, The Language of the classroom \cite{bellack1966language}
\item Ennis 1969, Logic in Teaching \cite{ennis1969logic}
\item Smith and Meux: Study in the logic of teaching \cite{Smith1970-SMIASO-13}
\item Hyman 1974: Vantage Points for Study \cite{hyman1968teaching}
\end{itemize}


\section*{Bellack Language of the Classroom}
\label{sec:org1d69eb8}

\subsection*{Chapter 2: Basic System For Analysis}
\label{sec:org24482e1}

\subsubsection*{Overview}
\label{sec:org171114b}
To analyze classroom discourse Bellack et al. developed a "system of categories devised to describe the verbal performance of teachers and students". Protocols of classroom discourse were analyzed in terms of these categories. Basically, and probably unbeknownst to them, they developed a programming language that describes classroom discourse. Keep in mind that their data is based off of transcripts of lectures, so their coding scheme naturally reflects that. In this section I'm going to explicate, in detail, their system. I'll start with a quick overview, then define the semantics of each of their terminals, and then dive deep into a few so you can see some examples and explore some of the mechanics. If you don't have time just read "The System", "Grammar", "Semantics"
\subsubsection*{The System}
\label{sec:org167b3de}
This system is heavily based on Wittgenstein's concept of a language game. The basic categories are:
\begin{itemize}
\item Pedagogical Moves \(\triangleq\) classroom discourse is viewed as a kind of "language game" (this is a technical term from Wittgenstein, specifically late-Wittgenstein, I can give you a run down later but suffice to say that the "door depends upon the hinge - Wittgenstein in \emph{On Certaintly}"). The basic unit of discourse is deemed a "pedagogical move" in this system. There are four types of pedagogical moves in the system:
\begin{enumerate}
\item Structuring
\item Soliciting
\item Responding
\item Reacting
\end{enumerate}
\item Teaching Cycles \(\triangleq\) Pedagogical moves occur in classroom discourse in certain cyclical patterns and combinations that are designated teach cycles.
\item Categories of Meaning \(\triangleq\) Four functionally different types of meaning are communicated by teachers and pupils in the classroom:
\begin{enumerate}
\item Substantive with associated
\item Substantive-logical meanings
\item instructional with associated
\item instructional-logical meanings
\end{enumerate}
\end{itemize}
Coding is done from the viewpoint of the observer, with pedagogical meaning inferred from the speaker's verbal behavior. 
\subsubsection*{Grammar}
\label{sec:org95e3007}
Each pedagogical move is coded as follows, I've added indices to follow the text: \\
\begin{verbatim}
Move = 1. Speaker 
       / 2. Type of Move 
       / 3. Substantive Meaning
       / 4. Substantive-Logical Meanings 
       / 5. Number of Lines in (3) or (4) 
       / 6. Instructional Meanings 
       / 7. Instructional-Logical Meanings 
       / 8. Number of lines in (6) or (7) 
\end{verbatim}

\subsubsection*{Semantics}
\label{sec:org5f2d277}

\begin{itemize}
\item \textbf{Semantic of Moves:}
\label{sec:org3bf13b5}
I'll elucidate the meaning of each move, I'll maintain the indices from the Grammar to avoid confusion. These will be brief definitions, for something more fleshed out and  with much more detail see the latter sections, specifically the "Categories of Meaning" Section:
\begin{enumerate}
\item \textbf{Speaker} \(\triangleq\) Indicates the source of the utterance, is one of:
\begin{enumerate}
\item \emph{Teacher} (T)
\item \emph{Pupil} (P)
\item \emph{Audio-Visual Device} (A)
\end{enumerate}
\item \textbf{Type of Pedagogical Move} \(\triangleq\) reference to function of move, there are 2 types each with sub-types:
\begin{enumerate}
\item Initiatory Moves
\begin{enumerate}
\item \emph{Structuring} (STR) \(\triangleq\) sets context for subsequent behavior by launching or halting-excluding interaction.
\item \emph{Soliciting} (SOL) \(\triangleq\) directly elicits a verbal, physical, or mental response; coded in terms of response expected.
\end{enumerate}
\end{enumerate}
\begin{enumerate}
\item Reflexive Moves
\begin{enumerate}
\item \emph{Responding} (RES) \(\triangleq\) fulfills expectation of solicitation; bears reciprocal relation only to solicitation.
\item \emph{Reacting} (REA) \(\triangleq\) modifies (by clarifying, synthesizing, expanding) and/or rates (positively or negatively); occasioned by previous move, but not directly elicited; reactions to more than one previous move coded in italics like this \emph{REA}
\end{enumerate}
\item \emph{Not Codable} (NOC) \(\triangleq\) serves as the \(\perp\) in their coding scheme.
\end{enumerate}
\item \textbf{Substantive Meaning} \(\triangleq\) reference to a subject matter topic
\item \textbf{Substantive-Logical Meaning} \(\triangleq\) reference to cognitive process involved in dealing with subject matter under study. 3 Main Types each with subtypes:
\begin{enumerate}
\item Analytic Process \(\triangleq\) Use of language or established rules of logic
\begin{enumerate}
\item \emph{Definining-Denotative} (DED) \(\triangleq\) object referent of term
\item \emph{Defining-Connotative} (DEC) \(\triangleq\) defining characteristics of class or term
\item \emph{Interpreting} (INT) \(\triangleq\) verbal equivalent of a statement, slogan, aphorism, or proverb
\end{enumerate}
\item Empirical Process \(\triangleq\) sense experience as criterion of truth
\begin{enumerate}
\item \emph{Fact-Stating} (FAC) \(\triangleq\) what is, was, or will be without explanation or evaluation.
\item \emph{Explaining} (XPL) \(\triangleq\) relation between objects, events, principles, conditional inference, cause-effect, explicit comparison-contrast, statement of principles, theories or laws
\end{enumerate}
\item Evaluative Process \(\triangleq\) set of criteria or value system as basis for verification
\begin{enumerate}
\item \emph{Opining} (OPN) \(\triangleq\) personal values for statement of policy, judgment or evaluation of event, idea, state of affairs, direct and indirect evaluation included
\item \emph{Justifying} (JUS) \(\triangleq\) reasons or argument for or against opinion or judgment
\end{enumerate}
\item \emph{Logical Process Not Clear} (NCL) \(\triangleq\) this serves as \(\perp\) for Substantive-Logical Meanings
\end{enumerate}
\item \textbf{Number of Lines in 3 and 4 above}
\item \textbf{Instructional Meanings} \(\triangleq\) reference to factors related to classroom management
\begin{enumerate}
\item \emph{Assignment} (ASG) \(\triangleq\) suggested or required student activity; reports, tests, readings etc.
\item \emph{Material} (MAT) \(\triangleq\) teaching aids and instructional devices
\item \emph{Person} (PER) \(\triangleq\) person as physical object or personal experiences
\item \emph{Procedure} (PRC) \(\triangleq\) Plan of activities or a course of action
\item \emph{Statement} (STA) \(\triangleq\) Verbal utterance, particularly the meaning validity, truth, or propriety of an utterance
\item \emph{Logical Process} (LOG) \(\triangleq\) Function of language or rule of logic; reference to definitions or arguments, but not presentation of such
\item \emph{Action-General} (ACT) \(\triangleq\) performance (vocal, non-vocal, cognitive, or emotional) the specific nature of which is uncertain or complex
\item \emph{Action-Vocal} (ACV) \(\triangleq\) physical qualities of vocal action
\item \emph{Action-Physical} (ACP) \(\triangleq\) physical movement or process
\item \emph{Action-Cognitive} (ACC) \(\triangleq\) Cognitive process, but not the language or logic of a specific utterance; thinking, knowing, understanding, or listening
\item \emph{Action-Emotional} (ACE) \(\triangleq\) emotion or feeling, but not expression of attitude or value
\item \emph{Language Mechanics} (LAM) \(\triangleq\) the rules of grammar and/or usage
\end{enumerate}
\item \textbf{Instructional-Logic Meaning} \(\triangleq\) reference to cognitive processes related to the distinctly didactic verbal moves in the instructional situation
\begin{enumerate}
\item Analytic Process: see (4) above
\item Empirical Process: see (4) above
\item Evaluative Process: includes the definitions in (4) above and:
\begin{enumerate}
\item Rating \(\triangleq\) reference to metacommunication; usually an evaluative reaction (REA)
\begin{enumerate}
\item \emph{Positive} (POS) \(\triangleq\) distinctly affirmative rating
\item \emph{Admitting} (ADM) \(\triangleq\) mild or equivocally positive rating
\item \emph{Repeating} (RPT) \(\triangleq\) implicit positive rating when statement (STA) is repeated by another speacker; also for SOL to repeat a vocal action (ACV)
\item \emph{Qualifying} (QAL) \(\triangleq\) explicit reservation stated in rating; exception
\item \emph{Not Admitting} (NAD) \(\triangleq\) rating that rejects by stating the contrary; a direct refutation
\item \emph{Negative} (NEG) \(\triangleq\) distinctly negative rating
\item \emph{Positive/Negative} (PON) \(\triangleq\) SOL requesting positive or negative rating
\item \emph{Admitting/Not Admitting} (AON) \(\triangleq\) SOL asking to permit or not permit specific action
\end{enumerate}
\item Extra-logical Process \(\triangleq\) SOL expecting physical action or when logical nature of verbal response cannot be determined. 
\begin{enumerate}
\item \emph{Performing} (PRF) \(\triangleq\) asking, demanding; explicit directive or imperative
\item \emph{Directing} (DIR) \(\triangleq\) SOL with or without stated alternatives; asking for directive, not permision for specific action
\item \emph{Extra-logical Process Not Clear} (NCL) \(\triangleq\) \(\perp\) for extra-logical process
\end{enumerate}
\end{enumerate}
\end{enumerate}
\item \textbf{Number of Lines in 6 and 7 above}
\end{enumerate}

\item The / Operator:
\label{sec:orgbfc5dc2}
The Moves constituents are syntactically conjoined into strings with the "/" operator in order: \\
\begin{equation}
   \(\frac{n \in \text{Move Constituent} \quad m \in \text{Move Constituent}}{n/m \in \text{Partial Move}}\)
\end{equation}
\item An Example Sentence
\label{sec:orgb0c2c3f}
coded pedagogical move is: \\
\begin{verbatim}
  T/STR/IMX/XPL/4/PRC/FAC/2
\end{verbatim}
The interpretation is as follows:
\begin{verbatim}
  T / STR / IMX / XPL /  4  / PRC / FAC / 2
 (1)/ (2) / (3) / (4) / (5) / (6) / (7) /(8)
\end{verbatim}
This translates to: A \emph{teacher} (1) makes a \emph{structuring} (2) move in which they \emph{explain} (4) something about  \emph{imports and exports} (3) for \emph{four} (5) lines of transcript and also states \emph{facts} (7) about class \emph{procedures} (6) for \emph{two} (8) lines of the transcript. 
\item Structuring (STR)
\label{sec:orgc10ee5d}
\begin{itemize}
\item \uline{Purpose}: \\
Structuring moves function to \emph{set the context} for subsequent behavior. Think of this as identifying, setting, or initializing the state.
\item \uline{Mechanics}: This is achieved by:
\begin{enumerate}
\item Launching or Halting or Exluding interactions bewteen teacher and pupils
\item Indicating the nature of the interaction in terms of the dimensions of time, agent, activity, topic and cognitive process, regulations, reasons, and instructional aid.
\end{enumerate}
\item \uline{Response}: Structuring moves do not elicit a response, are not direct responses and are not called out by anything in particular in the classroom
\item \uline{Examples}:
\begin{enumerate}
\item T/STR: All right, getting down to it now, I think international trade, then, or international economic relations, whatever you call it, is a field of study within economics which in many cases has been unfortunately divorced from or too far divorced from domestic trade because there are great similarities, and also there are some rather distinct differences.
\end{enumerate}
\end{itemize}
\item Soliciting (SOL)
\label{sec:org216cff1}
\begin{itemize}
\item \uline{Purpose}: \\
These are moves that tend to elicit an active verbal response, a cognitive response, or a physical response to the persons addressed.
\item \uline{Mechanics}: Soliciting moves are encoded in terms of the response expected rather than the solicitation itself
\item \uline{Examples}:
\begin{enumerate}
\item T/SOL: What are the factors of production?
\item P/SOL: May we keep our books open?
\item T/SOL: Pay attention to this!
\end{enumerate}
\end{itemize}
\item Responding (RES)
\label{sec:org8b5b6d8}
\begin{itemize}
\item \uline{Purpose}: \\
These moves are reciprocal to soliciting moves and only occur in relation to them. They function to fulfill the expectation of soliciting moves, and as such are reflexive.
\item \uline{Mechanics}: There can be no solicitation that is not intended to elicit a response, and no response that has not been directly elicited by a solicitation
\item \uline{Examples}:
\begin{enumerate}
\item T/SOl: What are the factors of production?
P/RES: Land, Labor, and capital.
\item T/SOL: What is exchange control
P/Res: I don't know
\end{enumerate}
\end{itemize}
\item Reacting (REA)
\label{sec:orgb101b66}
\begin{itemize}
\item Purpose: \\
These moves are occasioned by a structuring, soliciting, responding, or some prior move but are not directly elicited by them. These qualify the moves that preceded them, either by clarifying some point, synthesizing a new one, or expanding on a previous point.
\item \uline{Mechanics}: Preceding moves \emph{only} serve as the occasion for reaction moves.
\item \uline{Examples}:
\begin{enumerate}
\item T/REA: All right
\item T/REA: That's partly it
\item T/REA: Good. It limits specifically the number of items of one type or another which can come into this country. For example, we might decide that no more than one thousand German cars will be imported in any one calendar year. This is a specific quota which the government checks.
\end{enumerate}
\end{itemize}
\begin{itemize}
\item \uline{Special Modifiers}: When a reaction move is italicized (\emph{REA}); this denotes that the reaction is occasioned by more than a single move.
\begin{itemize}
\item \uline{Special Modifier Examples}:
\begin{enumerate}
\item T/REA: All of the instances of foreign investment that we have discussed here can be classified as either direct or portfolio types of investment.
\end{enumerate}
\end{itemize}
\end{itemize}
\end{itemize}

\subsubsection*{Teaching Cycles}
\label{sec:org4eb29e3}
Pedagogical moves occur in cyclical patterns and combinations, which are designated as \emph{teaching cycles}. A cycle begins with a structuring move or a solicitation that is not preceded by a structuring move. A cycle ends with the move that precedes a new cycle. Teaching cycles are coded \emph{only after} all moves have been coded. These define structuring and soliciting moves as \emph{initiatory} and responding and reacting moves as \emph{reflexive}. There are 21 types of teaching cycles possible. The first 12 are structure-initiated and the last 9 are initiated by soliciting moves. 

\begin{itemize}
\item Types of Teaching Cycles
\label{sec:org757c3c6}
\begin{enumerate}
\item STR
\item STR SOL
\item STR REA
\item STR REA REA \ldots
\item STR SOL RES
\item STR SOL RES RES \ldots
\item STR SOL REA
\item STR SOL REA REA \ldots
\item STR SOL RES REA
\item STR SOL RES REA REA \ldots
\item STR SOL RES REA RES \ldots
\item STR SOL RES REA RES \ldots REA \ldots
\item SOL
\item SOL RES
\item SOL RES RES \ldots
\item SOL REA
\item SOL REA REA \ldots
\item SOL RES REA
\item SOL RES REA REA \ldots
\item SOL RES REA RES \ldots
\item SOL RES REA RES \ldots REA \ldots
\end{enumerate}

Each type of teaching cycle represents a different pattern of pedagogical discourse. For example, Cycle 9, represents a pattern of discourse that is initiated by a structuring move and is followed by a solicitation; this solicitation then elicits a response that is the occasion for a reaction. Cycle 21 is initiated by a solicitation that elicits multiple responses which are in turn the occasion for multiple reactions. 
\item Purpose of Teaching Cycles:
\label{sec:org1c63983}
Teaching Cycles provide a way of describing pedagogical moves in relationship to each other. By utilizing the concept of teaching cycles it becomes possible to determine the extent to which solicitations elicit single or multiple responses or the regularity with which reactions follow responses. If a single pedagogical move may be compared to a move in chess or a single play in football. Then teaching cycles are an interrelated series of moves or plays, like a drive in football or a strategy in chess.
\end{itemize}
\subsubsection*{Categories of Meaning}
\label{sec:org85317f1}
There are four functionally different types of \emph{meaning} communicated in the classroom. \emph{substantive} with associated \emph{substantive-logical} meanings, \emph{instructional} with associated \emph{instructional-logical} meanings. Within each pedagogical move these four types of meaning are identified when they appear in the discourse
\begin{itemize}
\item \textbf{Substantive}
\label{sec:org5fbd053}
meanings refer to the subject matter under study by the class. For example, some previous example's substantive meanings referred to a teaching unit based on \emph{International Economic Problems} By James Calderwood. These meanings can then be \emph{atomized}, note that sub codings are possible as well, here are just a few examples:
\begin{enumerate}
\item \emph{Trade} (TRA) \(\triangleq\) refers to General discussions of trade; nature of trade in broad terms.
\begin{enumerate}
\item \emph{Trade} - \emph{Domestic and International} (TDI) \triangeq refers to Domestic trade compared and contrasted with international trade.
\end{enumerate}
\item \emph{Factors of Production and/or Specialization} (FSP) \(\triangleq\) refers to general discussion of factors of production; what they are; specialization etc.
\item \emph{Barriers to Trade} \(\triangleq\) General discussion of barriers, including policies directed toward maintaining or increasing barriers.
\item \emph{Not Trade} (NTR) \(\triangleq\) Discussion not about trade or economics
\item \emph{Promoting Free Trade} (PFT) \(\triangleq\) Discussions regarding the promotion of free trade
\item \emph{Barrer-Tariffs} (BAT) \(\triangleq\) Specific Discussions of tariff.
\end{enumerate}
\item \textbf{Substantive-logical meanings}
\label{sec:orgbd87497}
refer to the cognitive processes involved in dealing with the subject matter under study. Substantive-logical meanings are categorized under three generate headings:
\begin{itemize}
\item \uline{Analytic Process}:  Analytic statements are statements about the proposed use of language. Analytic statements are true by virtue of the meaning of the words of which the are composed (they are true \emph{a priori}). Or in other words these are true by an agreed upon set of rules and inferences. Statements like "All single Men are Bachelors". The book describes several sub-classifications here are just a few:
\begin{enumerate}
\item Defining-General (DEF) ::= To define in a general manner is to give the defining characteristic of a \emph{class} \emph{and} to give a specific example of an item within that class. DEF is also coded when the type of definition asked for or given is not clear. \\
Examples:
\begin{enumerate}
\item T: What is a barrier? Code: T/SOL/BAR/DEF/1/-/-/-
\item P: It's something that hinders trade. Code: P/RES/BAR/DEF/2/-/-/-
\end{enumerate}
\item Defining-Denotative (DED) ::= To define denotatively is to refer to the objects to which the term is applicable. A denotative definition cites the objects to which the term may correctly be applied, and these objects constitute the denotation of the term.
\item Defining-Connotative (DEC) ::= To define connotatively is to give the set of properties or characteristics that an object (abstract or concrete) must have for the term to be applicable. DEC refers to the defining characteristics of a given term. \\
Examples:
\begin{enumerate}
\item T: Now what do we mean by quotas? Code: T/SOL/BAQ/DEF/1/-/-/-
\item P: The government sets a special amount \\
of things that can come into the country\\
in one year, and no more can come in. Code: P/RES/BAQ/DEC/4/-/-/-
\end{enumerate}
\end{enumerate}

\item \uline{Empirical Process}: Empirical statements give information about the world, based on one's experience of it. These statements are distinguished by tests that are conducted in reference to one's experience. Some Examples:
\begin{enumerate}
\item Fact-Stating (FAC) ::= Giving an account, description or report on the current state of affairs. To State a fact is to state what is. Coding Example:
\begin{enumerate}
\item T: Now in 1934 \ldots in 1934 \ldots who was President? Code: T/SOL/PFT/\textit{FAC}/2/-/-/-
\item P: Roosevelt Code: P/RES/PFT/\textit{FAC}/1/-/-/-
\end{enumerate}
\item Explaining (XPL) ::= TO explain is to relate an object, event, action, or state of affairs to some other object, event, action, or state of affairs; or to show the relation between an event or state of affairs and a principle or generalization; or to state the relationships between principles or generalizations. For this document explanation and inference are taken to be identical. A statement is coded XPL when it concerns the \emph{effect} of some event or state of affairs on some other even or state of affairs; or when a statement provides \emph{reasons} for some event or state of affairs. Coding Examples:
\begin{enumerate}
\item T: What would happen if we raised the tariff on transistor radios? Code: T/SOL/BAT/XPL/2/-/-/-
\item P: Prices would go up. Code: P/RES/BAT/\textit{XPL}/1/-/-/-
\end{enumerate}
\end{enumerate}
\item \uline{Evaluative Process}: Evaluative statements are statements of grade, praise, blame, condemnation, or criticism. I'm only going to show the OPN example, there is also a Justification example (JUS) in the book:
\begin{enumerate}
\item Opining (OPN) ::= To make a statement of opinion, simply to opine. Includes statements of the nature: 1) What ought to be done or 2) fairness, worth, importance, or quality of an action, event, person, idea, plan, or policy. Coding Example:
\begin{enumerate}
\item P: I think the farmer is being exploited. Code: P/REA/BAT/\textit{OPN}/1/-/-/-
\end{enumerate}
\end{enumerate}
\end{itemize}
\item \textbf{Instructional Meanings}
\label{sec:org79b2c93}
These refer to conversations about classroom management, assignments, procedures etc. I'm going to omit these becasue they will not pertain to our explanatory objects and the book goes enumerates many of them in great detail.
\item \textbf{Instructional-Logical Meanings}
\label{sec:orgb3b580c}
Meanings that include those processes listed under substantive-logical meanings, but also refer to distinctly didactic verbal moves such as those involved in positive or negative rating and giving instructions. These include statements of qualitative judgment. Its easier to understand with examples. There are many of these but I'll only explicate a few to get the point across:
\begin{enumerate}
\item Positive (POS) \(\triangleq\) refers to distinctly affirmative rating, usually in a reaction to a statement.
\begin{enumerate}
\item T: Right! Code: T/REA/TDI/-/-/STA/\textit{POS}/1
\end{enumerate}
\item Admitting (ADM) \(\triangleq\) Hesitation of part of rater, mildy accepting or equivocally positive rating usually in reaction to a statement.
\begin{enumerate}
\item T: Mm-hmm. Code: T/REA/PFT/-/-/STA/\textit{ADM}/1
\end{enumerate}
\end{enumerate}
\end{itemize}
\section*{Explaining and Explanations Typology}
\label{sec:orgefb0efb}

\subsection*{Keys}
\label{sec:orgaee5330}
An explanation is divided up into "keys":
\begin{itemize}
\item Keys \(\triangleq\) A Key is part of the main explanation that must be explained for one to understand that which is to be explained
\end{itemize}
\subsection*{Links}
\label{sec:orgbd5d8a7}
One would then seek to "link" the discrete "keys" together: \\
\begin{description}
\item[{Links}] PAPER DOES NOT DEFINE
\end{description}
\subsection*{Rules}
\label{sec:org73982fd}
One would also define any rules that are pertinent to what is being explained
\begin{description}
\item[{Rules}] PAPER DOES NOT DEFINE
\end{description}
\subsection*{The Typology}
\label{sec:org89db2fc}
The Typology is formed by 3 categories of explanation:

\subsubsection*{The Interpretive}
\label{sec:org86aad10}
That which clarifies, exemplifies, or interprets the meaning of terms. Roughly speaks to "What is \ldots{} ?" 
\subsubsection*{The Descriptive}
\label{sec:org8597085}
That which describes a process or structure. Speaks to "How is ..? How does \ldots{}?" 
\subsubsection*{The "Reason Giving"}
\label{sec:orgb0a82bd}
That which offers reasons or causes. Speaks to the occurrence of a phenomenon (Why is \ldots{}?) \\

\subsubsection*{An Example:}
\label{sec:org59522b5}

\begin{itemize}
\item Interpretive Explanations:
\label{sec:org7ec6833}
\begin{itemize}
\item What are phyla?
\item What is a biome?
\item What is a fossil?
\item What is ecological succession?
\end{itemize}

\item Descriptive Explanations:
\label{sec:org34f00d4}
\begin{itemize}
\item Where does the energy of the living world come from?
\item How do streams come to be polluted?
\item How do environmental factors influence the number of plants and animals in a particular way
\item How are animal protected against the dangers of drying out
\end{itemize}

\item Reason-giving Explanations:
\label{sec:orgaabe2eb}
\begin{itemize}
\item Why are there no polar bears in the South Pole
\item Why is soil considered to be an ecosystem
\end{itemize}

I think that these roughly translate to our "Levels" in our coding scheme.
\end{itemize}

\section*{Gage et al Typology}
\label{sec:org513df32}
N.L. Gage's main programme was to identify objective variables in teaching lectures that could be used to form a \emph{standard} approach to lecturing. He compares the thrust of his work to that of physics; when physics hit a theoretical limit they often sought to surpass the limit by expanding and individuating constituents. For example, the Nucleus of an Atom is actually made up of subatomic particles, then they would seek to study the subatomic particles instead of the nucleus. This typology follows from Barak Rosenshines work See Chapter 9, pg 201, Research into Classroom Processes Westbury/Bellack. In this article Rosenshine develops the typology and then runs a user study to assess the correlative impact of each category on effective and ineffective lectures. So the typology merely serves as a way to classify, and differentiate good lectures from bad ones. This paper is meh, it doesn't define everything but the results are good although it can be hard to follow at times, especially the distinction between variables and categories - which I still don't understand.

\subsection*{Variables}
\label{sec:org2119a8f}
Rosenshine identifies 27 variables that were verbal, non-verbal, communicated by teachers and received by students. The variables were developed from research in four general areas:
\begin{enumerate}
\item Linguistics
\item Instructional Set
\item Experimental studies of instruction
\item Multivariate Studies of the Behavioral Correlates of Teacher Effectiveness
\end{enumerate}

\subsubsection*{Linguistic Categories}
\label{sec:org0f0cd8b}
Consists of 9 "categories"
\begin{enumerate}
\item Word Length
\item Total number of relevant words
\item Length and structure of independent clause unit
\item prepositional phrases
\item readability estimate \(\triangleq\) based on a multiple-regression formula developed by Flesch
\item personal references \(\triangleq\) counts of first and second person pronouns
\item negative sentences \(\triangleq\) counts of sentences containing "not" modifying the verb, noun or some similar negation
\item passive verbs \(\triangleq\) counts of independent or dependent clauses containing passive verbs
\item awkward and fragmented sentences \(\triangleq\) counts of sentences which depart from usual sentence construction or phrases which lack a subject of a verb but add information. e.g. "Now to foreign affairs"
\end{enumerate}
\subsubsection*{Instructional Set}
\label{sec:orgba7b0d2}
Consists of 2 categories
\begin{enumerate}
\item Structuring Sets \(\triangleq\) set that contains variables which are words or phrases which indicate that the speaker is attempting to clarify distinctions between new and previously learned material
\item Focusing or Arousing Sets \(\triangleq\) set that contains variables which might identify phrases designed to arouse or focus attention.
\end{enumerate}
\subsubsection*{Presentational Categories}
\label{sec:orgc55a2f4}
Consists of 9 categories
\begin{enumerate}
\item Use of rule-and-example Pattern
\item number of examples
\item organization of topics
\item use of enumeration
\item movement and gesture
\item breaks in speech
\item use of maps and chalkboard
\item rate of speech
\item repetition and redundancy
\end{enumerate}
\subsubsection*{Multivariate studies of teaching behaviors}
\label{sec:orga065f25}
Consists of 7 categories
\begin{enumerate}
\item Verbal Hostility
\item non-verbal affect
\item reference to pupil's interests
\item expansion of pupil's ideas
\item ratio of acceptance and praise to criticism
\item Conditional Words \(\triangleq\) counts of words such as "but", "however" and "although"
\item Explaining Links \(\triangleq\) prepositions and conjunctions which indicated the cause, result, means, or purposes of an event or idea.
\end{enumerate}

\subsection*{Takeaway}
\label{sec:orga738562}
We could co-opt a lot of rosenshines typology for our typology, a benefit of this would be that we would be in a position to co-opt his/her research on effective lectures to effective powerpoints/papers for algorithms.


\bibliographystyle{unsrt}
\bibliography{TheoryEdNotes}
\end{document}