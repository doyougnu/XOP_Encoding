% Created 2017-05-21 Sun 21:57
% Intended LaTeX compiler: pdflatex
\documentclass[10pt, letterpaper]{article}
\usepackage[utf8]{inputenc}
\usepackage[T1]{fontenc}
\usepackage{graphicx}
\usepackage{grffile}
\usepackage{longtable}
\usepackage{wrapfig}
\usepackage{rotating}
\usepackage[normalem]{ulem}
\usepackage{amsmath}
\usepackage{textcomp}
\usepackage{amssymb}
\usepackage{capt-of}
\usepackage{hyperref}
\usepackage[margin=1in]{geometry}
\usepackage{bussproofs}
\author{Jeffrey Young}
\date{May 21, 2017}
\title{Primer on Dan Hillman's Thesis}
\hypersetup{
 pdfauthor={Jeffrey Young},
 pdftitle={Primer on Dan Hillman's Thesis},
 pdfkeywords={},
 pdfsubject={},
 pdfcreator={Emacs 25.2.1 (Org mode 9.0.5)}, 
 pdflang={English}}
\begin{document}

\maketitle

\section*{Executive Summary}
\label{sec:orgc05307c}
\subsection*{Is Hillman's data suitable for our us:}
\label{sec:org802bb86}
Yes. I believe we will be able to make
use of his data, but we will not be able to say anything in regards to
Substantive Meaning using this data.

\subsection*{Is Hillman's typology more expressive or less expressive than Bellack's, should we us Hillman's or Bellack's:}
\label{sec:org6f1692c}
We should use Hillman's Typology, in many
ways it is a large improvement over Bellack's and his agreement rate is > 90\%.
We will need to modify his typology as he collapses the Substantive meaning
category so that it \emph{does not} need to be described \emph{for every} topic. 

\subsection*{What did Hillman learn from his data:}
\label{sec:org1516f70}
TBD, I ran out of time and didn't get to this.

\subsection*{Other considerations}
\label{sec:org9c7dff5}
You may want to look over the "Reference that will be useful" section as this
paper is well sources, and is a good branching point for other papers to
check out. I've highlighted some interesting ones.
\section*{Source Document}
\label{sec:org09f9064}
The source document, for this primer is located here:
\url{http://www.quahog.org/thesis/new.html}
\section*{Context}
\label{sec:orgce5ea3c}
This document is for the XOP Encoding project, participants include Eric
Walkingshaw and Jeffrey Young. This is the third such primer in the project,
the first one is a primer on DN-Theory, the second a primer on typologies of
explanation, specifically Bellack et al's typology. This document expands on
bellack et al's typology by summarizing the thesis of Dan Hillman. Dan Hillman
iterated on bellack's typology, changing and improving it in several ways, and
hence his typology is of interest to the XOP project. Dan Hillman was also kind
enough to provide the data he compiled with his typology for the XOP Project.
\section*{Orientation}
\label{sec:orgc325696}
This document is meant to explicate three things:
\begin{enumerate}
\item What are Hillman's changes to Bellack's typology and why
\item What is the nature of the data that Hillman compiled and it is useful to the
XOP project?
\item What did Hillman learn from his data, what were the conclusions of his
thesis?
\end{enumerate}

The document assumes you are familiar with Bellack et al's typology, and the
XOP Projects goals, progress, and current issues. If any of those is not the
case then please refer to prior documents to orientate yourself before reading
on.
\section*{Details on Hillman's Data}
\label{sec:org304ff17}

\subsection*{General Details of the Data}
\label{sec:orge3d96a8}
\begin{itemize}
\item n = \textasciitilde{}51,000
\item Data is composed of Face-To-Face (FTF) and Computer-Mediated-Communication
(CMC)
\item All data is from courses taught at New York School of Education between
1994 - 1995
\item Data was recorded from 2 classes:
\begin{enumerate}
\item Systems Analysis and Design
\item Database Management and Systems
\end{enumerate}
\item Both courses had FTF and CMC versions, and both versions taught the same
material
\end{itemize}
\subsection*{Details on the Transcription of the Data}
\label{sec:org89b7396}
\begin{itemize}
\item FTF courses were recorded on audio cassette, transcribed to word docs, then
munged into a database
\item Participants for CMC courses communicated, and were recorded, by Lotus Notes.
\item only interactions that encompassed \emph{the whole class} was included
\item interactions during breaks were not considered or included
\item Any small group discussions that took place were not considered or included
\item No one working on the transcript, editing, or the coding, had knowledge of
what was transcribed, which participant said it, or the purpose of the
research
\end{itemize}

\subsection*{Database and data information details}
\label{sec:org1d27a59}
\subsubsection*{Database Contains}
\label{sec:org13b995b}
\begin{itemize}
\item All text
\item Participants sex
\item Participants role (Teacher/Student)
\item Number of words in each sentence
\item Metadata of course (Course name, date etc.)
\end{itemize}
\subsubsection*{Data alterations}
\label{sec:orgb87bfab}
\begin{itemize}
\item All audio recordings were pre-pended with a character to denote the speaker, one of:
\begin{enumerate}
\item t \(\triangleq\) teacher
\item m \(\triangleq\) male student
\item f \(\triangleq\) female student
\end{enumerate}
\end{itemize}
\subsection*{Hillman's Problems with Bellack et al's typology}
\label{sec:org02ba95f}
Most of this is taken directly from Hillman's thesis:
\begin{enumerate}
\item Problem: Bellack's system fails to differentiate between a one-word
response and a one-liner response. This is consequent of Bellack et al's
decision to round any non-line utterance to length 1.

Effect: This gives unfair weight to utterances that are less than one-line
length, which distorts the differences between teacher and student
utterances (with the latter being inflated).
\item Problem: Structuring and Soliciting moves fail to capture monologues or exegesis

Effect: This constrains the systems unit of analysis, in fact, Hillman
found that studies which employed Bellack's system, and Bellack et al's
own data, have almost no monologues by the teacher, and are almost never
have adult student participants - only children.
\item Problem: The difference between Responding and Reacting moves is often
minimal and the two are easily interchangable, especially in asynchronous
communication (Bellack et al assumed synchronous communication e.g. a
conversation)

Effect: Superfluous encodings and noise in agreement rate
\item Problem: Substantive meanings fail to account for progressive levels of
meaning, which, in turn, make it difficult to code for any subject in
which the same idea or procedures are used at higher levels. For example,
in a math class one would learn multiplication or division not as an end
in and of itself, but as part of a larger process. Bellack's system cannot
account for this in a clean way.
\item Problem: Substantive Meanings are not abstracted from the course material
at all. In general, each substantive meaning is derived from the course
material, but if that material differs slightly than the meanings must
also change.

Effect: Comparing courses on the same material or topic becomes more
difficult
\item Problem: Instructional Meanings are similarly limited
\item Problem: Inclusion of an Audio-Video devices inflates its importance in
the classroom interaction to that of the participant. 

Effect: One cannot claim that if a teacher plays a move or audio snippet
as part of the lesson that the student is interacting with the content in
\emph{an observable} manner.
\item Problem: Bellack et al's system does not distinguish between differences
in students.

Effect: One cannot analyze the variable of sex in the data.
\item Problem: Bellack et al's system distinguishes between discussion that
occurs "as the result of an assignment", and intra-classroom discourse.

Effect: This excludes discourse which occurs from a teacher assigning work
\emph{and then} building on that assignment in class.
\end{enumerate}

\section*{Hillman's modifications to Bellack's Typology}
\label{sec:org7f63dee}
Hillman's Typology consists of three Tiers with the first being Purpose:
\subsection*{Purpose}
\label{sec:orged32213}
Pedagogical Moves are denoted by Purpose. Purpose has 7 Categories. Hillman
describes Purpose as a highway. In order to use a highway, one must get
directions to it, know the length of the journey, and the destination; these
would correspond to orienting utterances. When on the Highway, one moves
forward by the Lecturing purpose. A rest stop, to relax, recover, get gas or
food would be a Humanizing purpose. And finally an off-ramp would be an
Eliciting move, while an on-ramp would be a responding move. The exact
definitions of each are as follows:

\begin{enumerate}
\item Organizing: Similar to Structuring moves, organizing sentences do not
elicit a response and are not responses. Organizing sentences set an
agenda, organize a discussion or recitation, and function as a means to
get to other Purposes. Hillman describes them as functioning similar to an
on-ramp to a highway.

Ex. "In a minute I'll be handing you an overview of the course as well as
handouts for the first session." [Organising/Fact-Stating/Procedure]

\item Eliciting: Similar to Soliciting moves. Eliciting moves consist of
solicitations or explicit directives. They include all questions,
commands, imperatives, and requisitions. They are specifically designed to
cause interaction.

Ex. "Send me a Response to Response if you have any questions concerning
the basic forms creation process." [Eliciting/Performing/Procedure]

\item Responding: Responding moves combine reacting and responding moves from
Bellack's typology. They form a reciprocal relationship to \uline{any}
previously uttered move. In CMC courses, one may respond to a single word
of the electronic lecture, or to the whole lecture, thus a responding move
can "close" any previously uttered move, or moves. A responding move
concludes when sentences cease to serve the function of directly
responding to the previous moves.

Ex. "Yeah I hear, I hear." [Responding/Rating/Person]

\item Lecturing: Lecturing consists of talk about the course content that is
neither explaining a change in topic (Organising), soliciting a response
(Eliciting), nor Responding. Lecturing is differentiated from Responding
in that Responding is directly applicable to an Eliciting purpose. When
the Responding move has moved away from the purpose of merely answering
the Eliciting, it is then Lecturing. For example, suppose a student asked
a teacher what colour fire engines were. The immediate answer, "red" (or
"fluorescent yellow-green") would be a Responding sentence, but anything
beyond that, such as explaining why so many fire engines are red, would be
Lecturing.

Ex. "We went a little over tonight, but that's all right."
[Lecturing/Fact-Stating/Content]

\item Humanizing: Humanizing moves create an atmosphere conducive to interaction
by means of making student feel welcome jokes or small talk. Humanizing
moves' purpose is to make some feel at ease or maintain the relationship.
These moves are free from pedagogic content. This type also includes the
use of emoticons. Hillman speaks more to this latter use, it is not
represented here.

Ex. "You don't prefer to be called Jill?" [Humanising/Fact-Stating/Person]

\item Idling: This category has no analogy to Bellack's typology, rather this is
a pure addition. Idling sentences are sentences which are intelligible but
serve no pedagogical function, and unlike Humanizing sentences have no
defined goal e.g. to create a comforting environment. This type of move is
included because it is a way for the teacher to pause and collect their
thoughts.

Ex. "That you, you know when you, oh no, no, no." [Idling/Filler/Not Clear]

\item Not Clear: The bottom value for the Purpose tier. This is encoded when
words are unintelligible.
\end{enumerate}
\subsection*{Mechanism}
\label{sec:orgcc31fdd}
Mechanisms are similar to Instructional-Logical Meanings (Hillman uses
Instructional-Logical meanings, but I think he means Substantive-Logical),
they describe \emph{how} the subject of the sentence is being discussed. There are
9 sub-categories:
\begin{enumerate}
\item Fact-Stating: This is identical to Bellack et al.'s definition of
Fact-Stating

\item Explaining: This is a combination of Bellack et al.'s definitions of
Interpreting, and Explaining. This is used for sentences in which
clarification, definition, or rationale is explicitly given.

Ex. "It's a tool, and just like other tools (say automobiles, guns, and
chain saws), people can use it constructively and destructively, wisely
and wastefully." [Responding/Explaining/Content]

\item Opining: This is identical to Bellack et al.'s definition of Opining.

\item Performing: The Mechanism of Performing is similar to Bellack et al.'s
instructional-logical meanings of Performing and Directing, in which one
requests or expects an action to occur. Quite simply, Performing is the
process of telling someone to do something. 

Ex. "When you are finished with the student biography, pass them towards
the center, please." [Eliciting/Performing/Action]

\item Repeating: Hillman's own words are required for this one: 
The definition of Repeating is similar to that of Bellack et al. in which
one (in their case, presumably the teacher) repeats or rephrases what is
said (presumably by a student) as a way to indicate "an implicit
admitting" that the offered response was correct. An illustration from the
FTF transcripts:

Teacher: Now remember, the output of analysis is the input to--
[Eliciting/Performing/Content] Student: To design.
[Responding/Fact-Stating/Content]

Teacher: To design. What else do you need to know?
[Responding/Repeating/Content, Eliciting/Fact-Stating/Content]

My coding system, however, has broadened this definition to also include
repetition used for the purposes of setting context, which is not
necessarily an indication of agreement. In this example, the use of
repeating isn't merely to show one agrees (in the sense that the student
is correct), but it is also used for getting attention -- as a point of
focus -- to provide context for actions to follow. In this case, the
context is "You say 'design.' I acknowledge your answer, and use it as a
point of focus or context for my next move." This is also used in
asynchronous communications, such as USENET newsgroups or courses
delivered via CMC, in which text is quoted so that other participants will
understand what the respondent is talking about.

\item Rating: A combination of Bellack's Rating, and Acknowledgment. This
serves to appraise or acknowledge a participant's move.

Ex. "He's right." [Responding/Rating/Person]

\item Rhetorical Device: an Eliciting move that is \emph{not intended} to solicit a
response. This can be a graphic or some other prop used in the classroom.
A rhetorical device may be used in a series of Lecturing or Fact-Stating
moves for such a purpose. A rhetorical question would be an eliciting
purpose in conjunction with a rhetorical device mechanism.

Ex. See Hillman's thesis for a detailed example.

\item Filler: Analogous with the Idling Purpose, but for mechanisms.

\item Not Clear: The bottom value.
\end{enumerate}
\subsection*{Subject}
\label{sec:orgb8a84ee}
This tier describes \emph{what} is being discussed in the sentence, the content
being considered or statements \emph{about} something.

\begin{enumerate}
\item Person: The subject of a sentence is literally a person.

Ex. "I was pleased to have the opportunity to get to know you a bit
through my role as Client in the Systems Analysis course Case Study."
[Lecturing/Opining/Person]

\item Action: a person or object does something, this definition includes
Bellack et al.'s definition of Action.

Ex. "Those of you who do not have books, look on."
[Eliciting/Performing/Action]

\item Procedure: A subset of Action, in which one is told \emph{how} to do something,
rather than just \emph{to do} something.

Ex. "If you've already made a new replica of the User's Guide, please do
not replicate it further until the above mentioned posting."
[Eliciting/Performing/Procedure]

\item Content: This code is used for sentences related to course content. This
is Bellack's Substantive Meaning category, except that instead of defining
it in terms of the specific content of the subject, it is defined as being
\emph{on-topic}. If a sentence refers to the \emph{subject} of the course, then it
is \emph{on-topic}, and is Content.

Ex. "With relational, you basically retrieve multiple records at a single
time and the system decides how to access based on your call."
[Lecturing/Explaining/Content]

\item Supplies: The subject of the sentence deals with course material, teaching
aids, and devices, be they books, forms, projectors, computers etc..

Ex. "This tape is about three years old." [Lecturing/Fact-Stating/Supplies]

\item Not Clear: The bottom value for the Supply tier.
\end{enumerate}
\subsection*{Handling Combined moves}
\label{sec:org37cc4f5}
If a sentence encompasses more than one subject, it is coded as the highest
appropriate level. Thus, to use Bellack et al.'s example, a teacher directing
students to engage in some classroom procedure which required the use of
supplies would be coded as Procedure-Supplies. The rationale is that to have
the course, one must have students (Person). These students may be directed
(Action) to engage in some classroom Procedure (Procedure) which required the
use of Supplies (Supplies).

\section*{References that will be of use and why}
\label{sec:orgb3f7f70}
\begin{enumerate}
\item Paper: Weber, R. P. (1985). Basic content analysis. Beverly Hills: Sage
Publications.

Why: This paper defines and describes general steps to classify a text into
categories of content.

\item Paper: Atkinson, P. (1981). Inspecting classroom talk. In C. Adelman (Ed.),
Uttering, muttering: Collecting, using and reporting talk for social and
educational research, (pp. 98-113). London: Grant McIntyre Ltd. 

Why: This paper explains that, the ability to respond, and participate in
interactions, in synchronous communication, is made possible by the \emph{typing}
of utterances that immediately precede that one.

\item Paper: Sacks, H., Schegloff, E., \& Jefferson, G. (1974). A simplest
systematics for the organisation of turn-taking for conversation. Language,
50, 696-735.

Why: This paper expands on Atkinson's paper by describing interactions as
an adjacency pair.
\end{enumerate}


\section*{Other thoughts}
\label{sec:orgcd6c308}
\begin{itemize}
\item Bellack's system is synchronous i.e. based on discourse and inter-locution.
Computer based learning artifacts are asynchronous i.e. one sided exegeses.
The DSL we are trying to make is\ldots{}synchronous? or asynchronous, I'm
pretty sure we discussed this as synchronous.
\end{itemize}
\end{document}