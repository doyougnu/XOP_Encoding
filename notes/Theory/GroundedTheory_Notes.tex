% Created 2017-07-10 Mon 13:44
% Intended LaTeX compiler: pdflatex
\documentclass[10pt, letterpaper]{article}
\usepackage[utf8]{inputenc}
\usepackage[T1]{fontenc}
\usepackage{graphicx}
\usepackage{grffile}
\usepackage{longtable}
\usepackage{wrapfig}
\usepackage{rotating}
\usepackage[normalem]{ulem}
\usepackage{amsmath}
\usepackage{textcomp}
\usepackage{amssymb}
\usepackage{capt-of}
\usepackage{hyperref}
\usepackage[margin=1in]{geometry}
\usepackage{bussproofs}
\author{Jeffrey Young}
\date{June 28, 2017}
\title{Primer on Grounded Theory}
\hypersetup{
 pdfauthor={Jeffrey Young},
 pdftitle={Primer on Grounded Theory},
 pdfkeywords={},
 pdfsubject={},
 pdfcreator={Emacs 25.2.1 (Org mode 9.0.5)}, 
 pdflang={English}}
\begin{document}

\maketitle

\section*{Executive Summary}
\label{sec:org8d338cd}
\subsection*{What is Grounded Theory}
\label{sec:org2d6861c}
Grounded Theory is a research methodology that was created by sociologists in
the late 60s. It is qualitative, but still tries to be as systematic,
rigorous and sound as quantitative methods.

\subsection*{How will it be used for the project?}
\label{sec:org8662cd7}
We do not need to use the whole of grounded theory for our project because we
are not in need of a theory per se. Rather we need to use grounded theory's
coding ideas to generate a systematic, orthogonal database of codes that
describe the content in algorithm explanations

\subsection*{Why is it of use to XOP project?}
\label{sec:org5cf993e}
It roots the project in a well used, well studied, research method and
gives us a sound, systematic way to create the codes that are required to
analyze algorithm explanations.

\section*{Context}
\label{sec:org42b9ce7}
This document is for the XOP Encoding project, participants include Eric
Walkingshaw and Jeffrey Young. This is the fourth such primer in the project,
the first one is a primer on DN-Theory, the second a primer on typologies of
explanation, specifically Bellack et al's typology, the third on Dan Hillman's
PhD thesis. This document is meant to provide a decent summary for
understanding grounded theory. We are interested in grounded theory because it
offers a rigorous, scientific method for determining sets of tags or "codes"
from varied documents. In terms of our work, we believe that a comprehensive
set of codes, or rather, an ontology of algorithm-explanation descriptors will
serve as the basis for an XOP-based DSL.

\section*{Orientation}
\label{sec:orgfa6a69d}
This document is meant to explicate three things:
\begin{enumerate}
\item What is Grounded Theory?
\item Where did Grounded Theory come from?
\item How does one \emph{do} grounded theory?
\item What will grounded theory applied to the XOP project look like?
\end{enumerate}

\section*{Grounded Theory Terminology}
\label{sec:orga164814}
\begin{description}
\item[{memos}] are defined as a specialized type of written record, that contains
the products of analyses. These are basically thoughts a researcher
wants to record during an analysis phase. Memos are used for:
\begin{enumerate}
\item Open data exploration
\item Identifying/developing the properties and dimensions
concepts/categories
\item Elaborating the paradigm: the relationships between conditions,
actions/interactions, and consequences
\item Developing a story line
\end{enumerate}

\item[{Theoretical Sampling}] is a type of sampling that disregards typical
statistical considerations and instead focuses on filling in gaps in the
data based on a theory. For example, one might have data of apple growth
that has only been sampled during the day. A grounded theorist might say
"well we have no data on growth during the nighttime, so let's go collect
that". This is different than normal statistical sampling, which is just
focused on getting a \emph{representative} sample of some population.

\item[{Open Coding}] is the first stage of coding, in this stage a researcher
writes down terms, or tags that describe the data. The point
is to develop a sense of the salient categories that occur in
the data.

\item[{Axial Coding}] is the second stage of coding, in this stage the researcher
is trying to identify relationships between the tags that
were developed in the Open Coding phase. The goal of this
phase is to develop a \emph{coding paradigm} which is a
theoretical model that visually displays the inter
relationships between the codes

\item[{Selective Coding}] is the last stage of coding, in this stage the
researcher tries to identify one, or two central categories upon which
forms the basic and central phenomena for their theory. Then the
researcher tries to systematically relate the core categories to the
other categories or tags.

\item[{Saturation}] refers to the end point in coding. This occurs when a grounded
theorist adds new data (per Theoretic sampling) and no new
codes are identified in it. Some grounded theorists refer to
this as being as much a feeling as an identifiable end point.
\end{description}

\section*{What is Grounded Theory}
\label{sec:org372372f}
Grounded theory is a methodology for research that was created by sociological
researches in the mid 60s. The main idea is to generate, or discover theory
\emph{based on} data. In this sense, the theory is \emph{grounded} in the data. In the
original authors own words: "A grounded theory is one that is inductively
derived from the study of the phenomena it represents."\cite{corbin2014basics} 

\section*{Where did Grounded Theory Come From?}
\label{sec:orgf66eba1}
Grounded theory was developed by Strauss and Glaser working on sociological
health research in the 60s. It is qualitative, and is an attempt to show that
qualitative research can have a rigorous, sound, and useful methodology just
as quantitative research has.

\section*{What exactly is the method of Grounded Theory?}
\label{sec:orge80d70c}
The grounded theory method is robust and nuanced. From the sources I've read I
believe it is roughly as follows:
\begin{enumerate}
\item Pick a problem to research
\item Identify sources of data related to 1
\item Collect that data, whether it be interviews, public records, quantitative
data, personal letters
\item Open Code that data to develop tags, and some over-arching categories
\item Axially Code the tags and categories to identify and develop relationships
in the tags and categories
\item Selectively code the tags and categories to identify the central phenomena
in the data.
\item From Selective coding, the researcher should have developed a \emph{theory} that
is \emph{grounded} in the data.
\end{enumerate}

Central to grounded theory is that these steps are \uline{not} linear, rather they
happen simultaneously. The theorist is always moving back and forth between
these steps in order to develop a robust theory.

\section*{Central Tenets of Coding}
\label{sec:orgbb6ade5}

\subsection*{Open Coding}
\label{sec:org04e94c6}
See the definition above. In open coding the researcher should be asking
several questions as these code. These include "what is the core point of
this", "What would happen if x became y?", "how is this instance of this code
similar to other data that is coded similarly?". The latter is called
constant comparison.

\subsubsection*{Constant Comparison}
\label{sec:org1ef068e}
Constant Comparison is a technique that grounded researches use to make sure
that they are coding validly throughout the data. The central idea is that
when you code some data you think back to other times you've coded that data
and ask yourself if you are being consistent. If so, then good, if not then
you should revise the code or category in some way.

\subsubsection*{Saturation}
\label{sec:org8d13129}
One is done coding when introducing more data does not generate any new
tags. This is called \emph{saturation} by grounded theorists.

\subsection*{More Notes on Coding in General}
\label{sec:org8cae116}
Here is what some grounded theorists recommend one thinks about when coding. Use of the word text below could mean any qualitative data, not just text.

\subsubsection*{Ask}
\label{sec:org64ddd39}
Things to ask when coding a text are as follows:
\begin{itemize}
\item What is going on in this text?
\item What are the people saying? What are they actually trying to get at?
\item What are the people doing? What are they trying to accomplish?
\item What do these actions and statements take for granted? What is Assumed?
\item How do structure and context serve to support, maintain, impede or change these actions and statements?
\end{itemize}

\subsubsection*{Lofland Suggests, and guidelines of what codes can be about:}
\label{sec:org25366e0}
\begin{itemize}
\item Pay attention to acts, what are the brief events happening in the text?
\item Activities, what are the long events happening in the text?
\item Meanings, what concepts do people use to understand their world? What are their in vivo terms?
\item Participation, what are the involvement or adaptations happening to a setting?
\item Relationships, What are the relationships between people in the text? Familial, power \&c?
\item Settings, what are the contexts in which this text exists?
\end{itemize}

\subsubsection*{How to Identify Codes and Themes}
\label{sec:orgf59d046}
Source for most of this is Graham Gibbs youtube channel, and \cite{doi:10.1177/1525822X02239569}
\begin{itemize}
\item Use repetition, what is being said over and over again? What is the \emph{meaning} of that which is being repeated?
\item Indigenous Typologies (in vivo), what are the in-group terms that are being used? What is the vernacular someone uses to express they belong?
\item Metaphor and Analogies, look for metaphors and analogies in the text and see why they are being used, how they are used etc.
\item Transitions, look for transitions in the text to demarcate new topics and therefore new codes
\item Similarities and differences, use constant comparison, i.e. always ask how is this code similar or different to another use of the same code
\item Linguistic Connectives, things like because, before and after, suggest causal links. Look for Linguistic connectives to establish hierarchy in the code.
\item Omission, what is not being said?
\end{itemize}


\section*{How will we apply grounded theory to our project?}
\label{sec:org6e55e88}
The product of grounded theory is a theory that is grounded in the data. For
our use case we don't need a theory per se, rather we need a robust set of
identifiable, orthogonal codes that we can use to describe the information in
algorithm explanations. Using these codes we can construct our own Typology of
Explanation, or ontology of explanation (see previous primer's on Typologies).
Once we have a rigorous, system for describing the content of algorithm
explanations, and the way an explanation matures over the course of a
document, then we have the bones or foundation for a XOP DSL.

\section*{Sources}
\label{sec:org94995e3}
This primer was drawn from the following sources:
\begin{enumerate}
\item A youtube lecture series by Graham R. Gibbs: \url{https://www.youtube.com/watch?v=4SZDTp3\_New}
\item See references below: \cite{charmaz2006constructing}
\end{enumerate}

\bibliographystyle{unsrt}
\bibliography{TheoryEdNotes}
\end{document}