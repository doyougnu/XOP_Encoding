    \textbf{Notation} & Notation tags denote \emph{how }the aspect is actually expressed. If not otherwise stated, assumed to be text \\
    \hline
    %Tag section
    Cartoon & The input tag or aspect is represented in a drawn or animated graphic\\
    Code & The input tag or aspect is represented as a block of code from some programming language\\
    Mathematic & The input tag or aspect is represented using Mathematic formulae, variables, or equations\\
    PseudoCode & The input tag or aspect is represented in pseudocode; a programming-like language that is not a executable programming language\\
    Sequence & The input tag or aspect is being expressed in a ordered, bulleted or punctuated way\\
    Table & The input tag or aspect is explicitly displayed in a Table\\

    \\

    \textbf{Role} & each modifier takes any number of tags of any type as input and alters the meaning of the input tags in a specified way \\
    \hline
    % Tags
    Aside & Denotes that the input tags describe text that is not directly related to the scope the input tags refer to \\
    Caveat & Denotes that the purpose of the text is to further clarify a point, provide extra detail\\
    Meta & Denotes text that is about the current aspect but not directly explaining it\\
    Pedagogical & Denotes that the specific purpose of a statement is pedagogical in nature\\
    Related & Denotes that the input tags are substantially related to the parent aspect in some manner\\
    Review & Denotes that the purpose of the input tags is to provide a pedagogical review of the material to the reader\\
