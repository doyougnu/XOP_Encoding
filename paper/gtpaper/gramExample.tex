\begin{table}[H]
  \begin{tabular}{c p{0.8\linewidth} l}

    & Text & Coding \\
    \hline
    1 & \emph{2.2 Mergesort} & Algorithm \\
    \\

    2 & The algorithms that we consider in this section is based on a simple operation
    known as merging: combining two ordered arrays to make one larger ordered
    array. & Push Operation Description InVivo\\
    \\

    3 & This operation immediately lends itself to a simple recursive sort
    method known as mergesort: to sort an array, divide it into two halves, sort
    the two halves (recursively), and then merge the results. & Pop Description \\
    \\

    4 & \text{<cartoon of list>} & Cartoon \\
    \\

    5 & Mergesort guarantees to sort an array of N items in time proportional to N log
    N, no matter what the input. & Push Motivation Description \\
    \\

    6 & Its prime disadvantage is that it uses extra space proportional to N. & |
    Disadvantage Description \\
    \\

    7 & \emph{Abstract in-place merge.} & | Operation
  \end{tabular}
  \caption{Sample text and codings for beginning of MS03, italicized text are headers}
  \label{res:txt:ex}
\end{table}
  
   