\documentclass[sigconf]{acmart}

\usepackage{booktabs} % For formal tables
\usepackage{qtree}
\usepackage{amsmath,amssymb}
% \usepackage{cite}   % importing cite is throwing errors for some reason
\usepackage{color}
\usepackage{enumerate}
\usepackage{multicol}

% Get todos to render properly
\usepackage[obeyFinal]{easy-todo}

% Add package for ::= symbol, can't compile to pdf for some reason
% \usepackage{txfonts}

% Add package for well rendered quotations
\usepackage{dirtytalk}
\usepackage{hyperref}
\usepackage{graphicx}

\usepackage{lambda}

% correct bad hyphenation here
\hyphenation{op-tical net-works semi-conduc-tor}

% Copyright
%\setcopyright{none}
%\setcopyright{acmcopyright}
%\setcopyright{acmlicensed}
\setcopyright{rightsretained}
%\setcopyright{usgov}
%\setcopyright{usgovmixed}
%\setcopyright{cagov}
%\setcopyright{cagovmixed}


% DOI
\acmDOI{10.475/123_4}

% ISBN
\acmISBN{123-4567-24-567/08/06}

%Conference
\acmConference[Seattle'2017]{SIGCSE}{March 2017}{Seattle, Washington USA} 
\acmYear{2017}
\copyrightyear{2017}

\acmPrice{15.00}


\begin{document}
\title{A Domain Analysis of Data Structure and Algorithm Explanations in the Wild}


\author{Jeffrey Young} 
\affiliation{%
 \institution{Oregon State University}
 \department{School of EECS}
 \city{Corvallis} 
 \state{Oregon}
 \country{USA}}
\author{Eric Walkingshaw} 
\affiliation{%
 \institution{Oregon State University}
 \department{School of EECS}
 \city{Corvallis} 
 \state{Oregon}
 \country{USA}}
\todo{anonymize authors}


\begin{abstract}
Explanations of data structures and algorithms are complex interactions of
several notations, including natural language, mathematics, pseudocode, and
diagrams. Currently, such explanations are created ad hoc using a variety of
tools, and the resulting artifacts are static, reducing explanatory value. We
envision a domain-specific language for developing rich, interactive
explanations of data structures and algorithms. In this paper, we analyze this
domain to sketch requirements for our language. We build on an existing
pedagogic theory of explanation, which we adapt to a qualitative coding system
for explanation artifacts collected online. We show that explanations of
algorithms and data structures in the wild exhibit patterns predicted by the
pedagogic theory and derive insights for our language. This work is part of our
effort to develop the paradigm of explanation-oriented programming, which
shifts the focus of programming from computing results to producing rich
explanations of how those results were computed.
\end{abstract}

%
% The code below should be generated by the tool at
% http://dl.acm.org/ccs.cfm
% Please copy and paste the code instead of the example below. 
%
\todo{CCSXML}
\begin{CCSXML}
<ccs2012>
 <concept>
  <concept_id>10010520.10010553.10010562</concept_id>
  <concept_desc>Computer systems organization~Embedded systems</concept_desc>
  <concept_significance>500</concept_significance>
 </concept>
 <concept>
  <concept_id>10010520.10010575.10010755</concept_id>
  <concept_desc>Computer systems organization~Redundancy</concept_desc>
  <concept_significance>300</concept_significance>
 </concept>
 <concept>
  <concept_id>10010520.10010553.10010554</concept_id>
  <concept_desc>Computer systems organization~Robotics</concept_desc>
  <concept_significance>100</concept_significance>
 </concept>
 <concept>
  <concept_id>10003033.10003083.10003095</concept_id>
  <concept_desc>Networks~Network reliability</concept_desc>
  <concept_significance>100</concept_significance>
 </concept>
</ccs2012>  
\end{CCSXML}

\ccsdesc[500]{Computer systems organization~Embedded systems}
\ccsdesc[300]{Computer systems organization~Redundancy}
\ccsdesc{Computer systems organization~Robotics}
\ccsdesc[100]{Networks~Network reliability}


\todo{figure out keywords}
\keywords{ACM proceedings, \LaTeX, text tagging}

\maketitle

\begin{abstract}
  Explanations of data structures and algorithms are complex interactions of
  several notations, including natural language, mathematics, pseudocode, and
  diagrams. Currently, such explanations are created ad hoc using a variety of
  tools, and the resulting artifacts are static, reducing explanatory value. We
  envision a domain-specific language for developing rich, interactive
  explanations of data structures and algorithms. In this paper, we analyze this
  domain to sketch requirements for our language. We utilize research methods
  common in sociological work, which we adapt to form a qualitative coding
  system for explanation artifacts collected online. We show that explanations
  of algorithms and data structures in the wild exhibit patterns predicted by
  the pedagogic theory and derive insights for our language. This work is part
  of our effort to develop the paradigm of explanation-oriented programming,
  which shifts the focus of programming from computing results to producing rich
  explanations of how those results were computed.
\end{abstract}

\section{Introduction}
\label{sec:intro}

Data structures and algorithms are at the heart of computer science and must be
explained to each new generation of students. How can we do this effectively?


In this paper, we focus on the \emph{artifacts} that constitute or support
explanations of data structures and algorithms (hereafter just ``algorithms''),
which can be shared and reused.
%
For verbal explanations, such as a lecture, the supporting artifact might be
the associated slides. For written explanations, the artifact is the
explanation as a whole, including the text and any supporting figures.
%
Explanation artifacts associated with algorithms are interesting because they
typically present a complex interaction among many different notations,
including natural language, mathematics, pseudocode, executable code, various
kinds of diagrams, animations, and more.


Currently, explanation artifacts for algorithms are created ad hoc using a
variety of tools and techniques, and the resulting explanations tend to be
static, reducing their explanatory value.
%
Although there has been a substantial amount of work on algorithm visualization~
\cite{Gloor92,Gloor97,HDS02, shaffer2010algorithm, HANSEN2002291, KANN1997223},
and tools exist for creating these kinds of supporting artifacts, there is no
good solution for creating integrated, multi-notational explanations as a whole.
Similarly, although some algorithm visualization tools provide a means for the
student to tweak the parameters or inputs to an algorithm to generate new
visualizations, they do not support creating cohesive interactive explanations
that correspondingly modify the surrounding explanation or that allow the
student to respond to or query the explanation in other ways.
%
To fill this gap, we envision a \emph{domain-specific language} (DSL) that
supports the creation of rich, interactive, multi-notational artifacts for
explaining algorithms.
%
The development of this DSL is part of a larger effort to explore the new
paradigm of \emph{explanation-oriented programming}, briefly described in
Section~ \ref{sec:back:xop}.


The intended users of the envisioned DSL are CS educators who want to create
\emph{interactive artifacts} to support the explanation of algorithms. These
users are experts on the corresponding algorithms, and also trained and skilled
programmers. The produced explanation artifacts might supplement a lecture or
be posted to a web page as a self-contained (textual and graphical)
explanation.
%
The DSL should support pedagogical methods directly through built-in
abstractions and language constructs. It should also support a variety of forms
of student interaction. For example, teachers should be able to define
equivalence relations enabling users to automatically generate variant
explanations~\cite{EW13jvlc}, to build in specific responses to anticipated
questions, and to provide explanations at multiple levels of abstraction.


This paper represents a formative step toward this vision. We conduct a
\emph{qualitative analysis} of our domain in order to determine the form and
content of the explanation artifacts that educators are already creating.
%
We base our analysis on an established pedagogical theory in order to better
understand how well existing artifacts support the pedagogical process of
explaining a complex topic.


More specifically, we collect 30 explanation artifacts from the internet,
consisting of slide decks and lecture notes that explain two algorithms and one
data structure commonly covered in undergraduate computer science courses:
Dijkstra's shortest path algorithm~\cite[pp.~137--142]{KT06}, merge sort
\cite[210--214]{KT06}, and AVL trees \cite[pp.~458--475]{KnuthArt3}.
%
We analyze these artifacts by applying a classic pedagogical theory by Bellack
et al.~\cite{bellack1966language} that describes the patterns of language used
in the process of teaching. Bellack et al.\ define a typology for coding
transcripts of teacher and student verbalizations during teaching. An overview
of the typology is given in Section~\ref{sec:back:typ}.


This paper makes the following contributions:
%
\begin{enumerate}[C1.]

\item \label{contrib:codes}
%
We adapt Bellack et al.'s typology for analyzing \emph{explanation activities}
in the form of classroom discourse to support analyzing \emph{explanation
artifacts} for algorithms in the form of lecture notes and slides
(Section~\ref{sec:exp:typ}).

\item \label{contrib:data}
%
We provide a coded qualitative data set of explanation artifacts, using the
system defined in C\ref{contrib:codes} applied to our sample of 30 collected
explanation artifacts (Section~\ref{sec:exp:data}).

\item \label{contrib:valid}
%
We note that conceptual units called \emph{teaching cycles}, described in the
pedagogy literature, can be observed in the collected artifacts. This suggests
that the adaptation of Bellack's system to artifacts is valid and that similar
teaching strategies are used in explanation artifacts as in verbal explanations
(Section~\ref{sec:res:cycles}).

\item \label{contrib:analysis}
%
We further analyze the data set to motivate the need for a DSL and to extract
insights that will be useful for the design of such a DSL
(Section~\ref{sec:res:analysis}). For example, we note that the artifacts do
not include any steps coded with the \emph{reacting} or \emph{responding}
pedagogical moves defined by Bellack et al.'s typology, reflecting the fact
that static artifacts treat the learner as an information sink with no recourse
for query or response.

\item \label{contrib:dsl}
%
We describe how Bellack et al.'s theory can directly provide a semantics basis
for a DSL and argue for the advantages of such an approach (Section~\ref{sec:res:dsl}).
%
\end{enumerate}

\noindent

\section{Background}
\label{sec:back}

\begin{figure}
\Tree[.IP [.NP [.Det \textit{the} ]
               [.N\1 [.N \textit{package} ]]]
          [.I\1 [.I \textsc{3sg.Pres} ]
                [.VP [.V\1 [.V \textit{is} ]
                           [.AP [.Deg \textit{really} ]
                                [.A\1 [.A \textit{simple} ]
                                      \qroof{\textit{to use}}.CP ]]]]]]
\end{figure}

\bibliography{XOPbib.bib,eric.bib}
\bibliographystyle{ACM-Reference-Format}

\end{document}
