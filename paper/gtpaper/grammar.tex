\begin{table}[h!]
  \centering
  \begin{tabular}{c p{1.8\linewidth}}
    \textbf{Tag} & \textbf{Description} \\
    \hline
    \hline


    \textbf{Structure} &  Structure tags control how the context in the artifact is set \\
    \hline
    $\rightarrow$ & Push: takes one content tag and sets the context to the conjunction of the input tag and parent context \\
    $\leftarrow$ & Pop: takes any number of action or expression tags as input, sets context up one level, and then applies input tags\\
    $|$ & Swap: takes one content tag as input, synonymous with the sequence $\leftarrow, \rightarrow \text{<input tag>}$ \\
    $\twoheadleftarrow$ & Return: takes no input, pops until top level is reached \\

    
    \textbf{Content} & Content tags denote \emph{what} is being discussed in a snippet of text \\
    \hline
    %Tag section
    Advantages & The text is discussing the pros, the upside or the advantages of the parent context \\
    Algorithm & The text is discussing an algorithm in general \\
    Application &  The text is discussing the use cases, or applications of the parent context\\
    Class & Denotes the explicit discussion of a group, set or class of some thing\\
    Complexity & The text is discussing the computational complexity of the parent context\\
    Condition & The text is discussing a condition that whatever the context was set to has, that must be satisfied\\
    Constituent & The text is discussing some constituent part of the parent context\\
    Data Structure & The text is discussing a Data Structure, this is an analog to the Algorithm code\\
    Design & The text is talking about the design, or design considerations of the parent context\\
    Disadvantages & The text is discussing the downsides, the cons or shortfalls of the parent context\\
    Goal & The text is discussing the goal, the end game, that which is the desired outcome, of the parent context\\
    History & The text is discussing the history of the parent context\\
    Implementation & The text is discussing implementation details of the parent context\\
    Motivation & The text is discussing the motivation for the parent context\\
    Operation & The text is discussing an operation that is a requisite and central part of the parent context\\
    Problem & The text is discussing a problem that may or may not be solved later in the document\\
    Property & The text is discussing some property of the parent context context\\
    Solution & The text is discussing a solution to some prior introduced problem, which is typically in the parent context\\
    State & The text is now discussing something related to the state or the state of the parent context\\

    
    \textbf{Actions} & Action tags denote \emph{how} the document or author is discussing the context\\
    \hline
    %Tag section
    Abstraction & The text is abstracting or generalizing that which the context is set to\\
    Assumption & The text is giving the reader or telling the reader an assumption about the context \\
    Base Case & The text is giving an explicit base case in an inductive procedure in relation to the context\\
    Cases & The text is breaking down the context into chunks of information, or cases.\\
    Comment & A dummy code whose use is just to provide a binding for an Aside, Caveat, or Meta modifier\\
    Conclusion & The text is making a conclusion about the context\\
    Contrast &  The text is contrasting the context with the parent node \\
    Definition & The text is defining some term about the context\\
    Derivation & The text is making a derivation about something in relation to the context\\
    Description & This code is the most general Action code. It denotes that the text is describing the context in some manner.\\
    Example & The text is giving, or providing an example in relation to the context\\
    Implication & The text is giving an implication about the context. This could be anything that fits the logical connective if..then..else..\\
    In Vivo & The text is defining a term that practitioners of the context would be familiar with\\
    Legend & The text is giving a legend to understand something\\
    Observation & The text making a general observation about the context\\
    Outline & The text is giving a bulleted list of the content the document will go through\\
    Proof & The text giving a mathematic or logical proof about the context\\
    Proposal & The text suggesting a path forward.\\
    Solicitation & The text is explicitly asking something of the reader\\
    Summary & This code denotes a concluding block of text that summarizes the previous contexts\\

    
    \textbf{Expression} & Expression tags denote \emph{how }the content is actually expressed. If not otherwise stated, assumed to be text \\
    \hline
    %Tag section
    Cartoon & The content is represented in a drawn or animated graphic\\
    Code & The content is represented as a block of code from some programming language\\
    Mathematic & The content is represented using Mathematic formulae, variables, or equations\\
    PseudoCode & The content is represented in pseudocode; a programming-like language that is not a executable programming language\\
    Sequence & The content is being expressed in a ordered, bullted or punctuated way\\
    Table & The content is explicitly displayed in a Table\\


    \textbf{Modifiers} & each modifier takes any number of tags of any type as input and alters the meaning of the input tags in a specified way \\
    \hline
    % Tags
    Aside & Denotes that the input tags describe text that is not directly related to the scope the input tags refer to \\
    Caveat & Denotes that the purpose of the text is to further clarify a point, provide extra detail\\
    Meta & Denotes text that is about the current context but not directly explaining it\\
    Pedagogical & Denotes that the specific purpose of a statement is pedagogical in nature\\
    Related & Denotes that the input tags are substantially related to the parent context in some manner\\
    Review & Denotes that the purpose of the input tags is to provide a pedagogical review of the material to the reader\\
  \end{tabular}
  \caption{Grammar for Coding System}
  \label{res:tbl:grmr}
\end{table}
